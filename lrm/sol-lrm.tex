\documentclass[letterpaper,12pt]{article}

%% Use the option review to obtain double line spacing
%% \documentclass[preprint,review,12pt]{elsarticle}

%enumitem to itemize nicely
\usepackage{enumitem}

%source code highlighting
\usepackage{listings}

\lstset{
	breaklines=true
}

\begin{document}

%% Title, authors and addresses

\title{{\small better call} {\Huge \textbf{SOL}}\\
    \begin{center}{SHAPE ORIENTED LANGUAGE REFERENCE MANUAL}\end{center}
}


\author{Aditya Naraynamoorthy\\
    \texttt{an2753}
    \and
    Erik Dyer\\
    \texttt{ead2174}
    \and
    Gergana Alteva\\
    \texttt{gla2112}
    \and
    Kunal Baweja\\
    \texttt{kb2896}}

%% Generate the title
\maketitle

%% Table of contents
\tableofcontents

\section{Introduction}
%% Text of abstract
SOL is a simple language that allows programmers to create 2D animations with ease. Programmers will have the ability to define and create objects, known as shapes, and dictate where they appear, and how they move. As a lightweight object-oriented language, SOL allows for unlimited design opportunities and eases the burden of animation. In addition, SOL’s simplicity saves programmers the trouble of learning complicated third-party animation tools, without sacrificing control over behavior of objects.
\par

\section{Conventions}
The following conventions are followed throughout this SOL Reference Manual.

    \begin{enumerate}
        \itemsep0em
        \item \texttt{literal} - Fixed space font for literals such as commands, functions,\\
        \hspace*{4.4em} keywords, and programming language structures.
        
        \item \textit{variable} - The variables for SOL progamming language and words or\\
        \hspace*{4.4em} concept being defined are denoted in italics.
    \end{enumerate}

\section{Lexical Conventions}

This section describes the complete lexical conventions followed for a syntactically correct SOL program, forming various parts of the language.

    \subsection{\textit{Comments}}
    Comments in SOL start with character sequence \texttt{/*} and end at character sequence \texttt{*/}. They may extend over multiple lines and all characters following \texttt{/*} are ignored until an ending \texttt{*/} is encountered.

    \subsection{\textit{Identifiers}}
    In SOL, an identifier is a sequence of characters from the set of english alphabet, arabic numerals and underscore (\_). The first character cannot be a digit. Identifiers are case sensitive. Identifiers cannot be any of the reserved keywords mentioned in section \ref{keywords}.

    \subsection{\textit{Keywords}} \label{keywords}
    Keywords in SOL include data types, built-in functions, and control statements, and may not be used as identifiers as they are reserved.

        \begin{center}
            \begin{tabular}{ |c|c|c|c| } 
            \hline
                int     & if            & main          & shape \\ 
                float   & while         & setFramerate  & parent\\ 
                char    & func          & translate     & extends\\
                string  & construct     & rotate        & \\
                        & return        & render        & \\
                        &               & wait          & \\
                        &               & drawPoint     & \\
                        &               & drawCurve     & \\
                        &               & print         & \\
                        &               & length        & \\
                        &               & consolePrint  & \\
            \hline
            \end{tabular}
        \end{center}

    \subsection{\textit{Integer Constants}}
    A sequence of one or more digits representing a number in \textit{base-10}\\
    Eg: \texttt{1234}

    \subsection{\textit{Float Constants}}
    Similar to an integer, a float has an \textit{integer}, a decimal point (\texttt{.}), and a fractional part. Both the integer and fractional part are a sequence of one or more digits.\\
    Eg: \texttt{0.55  10.2}

    \subsection{\textit{Character Constants}}
    An ASCII character within single quotation marks.\\
    Eg: \texttt{'x' 'a'}

    \subsection{\textit{Escape Sequences}}
    The following are special characters represented by escape sequences.
        \begin{center}
            \begin{tabular}{ |c|c| }
            \hline
                \textbf{Name}   & \textbf{Escape}\\
                \hline
                newline         & \textbackslash n\\
                tab             & \textbackslash t\\
                backslash       & \textbackslash \textbackslash\\
                single quote    & \textbackslash '\\
                double quote    & \textbackslash "\\
                ASCII NUL character & \textbackslash 0\\
            \hline
            \end{tabular}
        \end{center}

    \subsection{\textit{String constants}}
    A SOL \textit{string} is a series of \textit{characters} within double quotation marks. Its type is an array (defined in ) of characters. The compiler places a null byte (\texttt{\textbackslash0}) at the end of a string literal to mark its end.\\
    Eg: \texttt{“cat”}

    \subsection{\textit{Operators}}
    SOL has mainly three categories of operators defined below:

        \subsubsection{\textit{Assignment Operator}}
        An \textit{assignment operator} is denoted by the \texttt{=} symbol having a variable identifier to its left and a valid expression on its right. The \textit{assignment operator} of the expression on the right to the variable on the left.

        \subsubsection{\textit{Arithmetic Operators}}
        SOL has following \textit{binary arithmetic operators}. A \textit{binary arithmetic operator} operates on two \textit{arithmetic expressions} specified before and after the operator respectively. The said expressions must be of type \textit{int} or \textit{float}

            \begin{center}
                \begin{tabular}{ |c|c| }
                \hline
                    \textbf{Operator} & \textbf{Definition}
                    \hline
                    +   &   Addition\\
                    -   &   Subtraction\\
                    *   &   Multiplication\\
                    /   &   Division\\
                    \%  &   Modulo\\
                \hline
                \end{tabular}
            \end{center}

        \subsubsection{\textit{Comparison Operators}}
        The comparison operators are binary operators for comparing values of operands defined as expressions.
            \begin{center}
                \begin{tabular}{ |c|c| }
                \hline
                    \textbf{Operator} & \textbf{Definition}
                    \hline
                    ==  & Equality \\
                    !=  & Not Equals \\
                    <   & Less than \\
                    >   & Greater than \\
                    <=  & Less than or equals \\
                    >=  & Greter than or equals \\
                \hline
                \end{tabular}
            \end{center} 

        \subsection{\textit{Array}} \label{array}
\end{document}

%%
%% End of file `elsarticle-template-1-num.tex'.