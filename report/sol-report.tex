%%Author: Javangers Team

\documentclass[letterpaper,12pt]{report}

%% page layouts
\usepackage[margin=0.8in]{geometry}

%% appendices
\usepackage[toc,page]{appendix}

%enumitem to itemize nicely
\usepackage{enumitem}

%source code highlighting
\usepackage{listings}
\usepackage{lstautogobble}
\usepackage{xcolor}

\usepackage[hidelinks]{hyperref}
\hypersetup{
    linktoc=all,     %sections and subsections linked
    urlcolor=blue
}

% display backtick
\usepackage{upquote}

% import images
\usepackage{graphicx}

\definecolor{lightgray}{rgb}{.95,.95,.95}

% gnu make listing
\lstdefinestyle{bash}{
   language=bash,
   keywordstyle=\bfseries\color{green!40!black},
   commentstyle=\itshape\color{purple!40!black},
   identifierstyle=\color{blue},
   extendedchars=true,
   basicstyle=\ttfamily,
   showstringspaces=false,
   showspaces=false,
   breaklines=true,
   showtabs=false,
   frame=single,
   stringstyle=\color{red},
   backgroundcolor=\color{lightgray},
   autogobble=true,
   aboveskip=-4pt
}

% gnu make listing
\lstdefinestyle{make}{
   language=[gnu]make,
   keywordstyle=\bfseries\color{green!40!black},
   commentstyle=\itshape\color{purple!40!black},
   identifierstyle=\color{blue},
   extendedchars=true,
   basicstyle=\ttfamily,
   showstringspaces=false,
   showspaces=false,
   breaklines=true,
   showtabs=false,
   frame=single,
   stringstyle=\color{red},
   backgroundcolor=\color{lightgray},
   autogobble=true,
   aboveskip=-4pt
}

% ocaml listing
\lstdefinestyle{caml}{
   language=[Objective]Caml,
   keywordstyle=\bfseries\color{green!40!black},
   commentstyle=\itshape\color{purple!40!black},
   identifierstyle=\color{blue},
   extendedchars=true,
   basicstyle=\ttfamily,
   showstringspaces=false,
   showspaces=false,
   breaklines=true,
   showtabs=false,
   frame=single,
   stringstyle=\color{red},
   backgroundcolor=\color{lightgray},
   autogobble=true,
   aboveskip=-4pt
}

% settings for SOL code listings
\lstdefinestyle{sol}{
   language=Java,
   morekeywords={func, construct, shape, string},
   keywordstyle=\bfseries\color{green!40!black},
   commentstyle=\itshape\color{purple!40!black},
   identifierstyle=\color{blue},
   extendedchars=true,
   basicstyle=\ttfamily,
   showstringspaces=false,
   showspaces=false,
   breaklines=true,
   showtabs=false,
   frame=single,
   stringstyle=\color{red},
   backgroundcolor=\color{lightgray},
   autogobble=true,
   aboveskip=-10pt
}

\begin{document}

%% Title, authors and addresses
\title{{\small better call} {\Huge \textbf{SOL}}\\
    \begin{center}{SHAPE ORIENTED LANGUAGE FINAL REPORT}\end{center}
}

\author{
\begin{tabular}{ lc lc lc }
Aditya Narayanamoorthy & \texttt{an2753}  & \textit{Language Guru}    \\
Erik Dyer              & \texttt{ead2174} & \textit{System Architect} \\
Gergana Alteva         & \texttt{gla2112} & \textit{Program Manager}  \\
Kunal Baweja           & \texttt{kb2896}  & \textit{Tester}
\end{tabular}
}

%% Generate the title
\maketitle

%% Table of contents
\tableofcontents{}

\chapter{Introduction}
  SOL (Shape Oriented Language), is a domain specific programming language that allows programmers to create 2D animations with ease, through an object-oriented approach. Engineers, programmers, scientists and designers, through SOL, have the ability to define and create objects, known as Shapes, and dictate their appearance and movements on the screen. SOL\'s simplicity saves developers the trouble of learning complicated third-party animation tools, without sacrificing control over behavior of objects. It compiles into LLVM IR bytecode, making it adaptable across different architectures. The produced LLVM IR bytecode can be translated further into assembly code and linked statically against a predefined library before compiling down into an executable for a specific architecture using the LLVM compiler and GCC compilers respectively. The predefined library used by SOL for graphic rendering has been built on top of SDL2, that abstracts away the lower level details for drawing and animating objects.

  \section{Background}
  SOL takes its inspiration from the ease of programming in \textit{object-oriented paradigm} and the complexity of existing libraries/solutions for rendering graphics. It attempts to combine both of them and provide a language which allows developers to organize graphics into a collection of \textit{Shapes} (much like objects) which can be easily defined, created and interacted with to produce powerful images and/or animations at a fast paced development.\\

  SOL is commonly used to model various types of scientific data, but it can also be applicable in other domains, such as:
  \begin{enumerate}
    \itemsep 0em
    \item Drawing engineering models
    \item Data visualization
    \item Bored college students making funny memes
    \item Entertaining animations
  \end{enumerate}

\chapter{Language Tutorial}
  This chapter provides a simple language tutorial to guide programmers towards creating their first SOL program! It has been divided into three parts:
  \begin{enumerate}
    \item \textbf{Environment setup} - guide to setting up the SOL compiler and development environment
    \item \textbf{Language Quick Tour} - a bried descriptioin of basic components of language.
    \item \textbf{Sample Program} - create and rotate a line about its midpoint
  \end{enumerate}

  \section{Environment setup}
  SOL compiler has been developed and tested on Ubuntu 15.04 and Ubuntu 16.04 environments, and capable of supporting others as well. It can work on multiple architecture as the LLVM IR bytecode can further be compiled into architecture specific executables. The environment setup tutorial assumes you have the latest version of SOL compile source code downloaded from our repository.

    \subsection{Building SOL compiler (Ubuntu 15.04 or later version)}
    \begin{enumerate}
      \item SOL compiler is written in OCaml. Download the source code from our github repository: \colorbox{lightgray}{{\href{https://github.com/bawejakunal/sol}{https://github.com/bawejakunal/sol}}}

      \item Download and run the bash script below, which installs the latest compatible version of OCaml compiler and opam for *-nix OS environments.\\
      \colorbox{lightgray}{{\href{https://raw.githubusercontent.com/ocaml/ocaml-ci-scripts/master/.travis-ocaml.sh}{https://raw.githubusercontent.com/ocaml/ocaml-ci-scripts/master/.travis-ocaml.sh}}}

      \item Configure the opam environment for importing Ocaml packages\\
      \colorbox{lightgray}{\lstinline[style=bash]{eval `opam config env`}}

      \item Install LLVM compiler toolchain and ocaml bindings, for the SOL compiler\\
      \colorbox{lightgray}{\lstinline[style=bash]{./install-llvm.sh}}

      \item Download and Install SDL2 and SDL2\_gfx libraries which are used by our predefined static library for graphics rendering.\\
        \colorbox{lightgray}{\lstinline[style=bash]{wget} \href{http://www.ferzkopp.net/Software/SDL2\_gfx/SDL2\_gfx-1.0.3.tar.gz}{http://www.ferzkopp.net/Software/SDL2\_gfx/SDL2\_gfx-1.0.3.tar.gz}}\\
        \colorbox{lightgray}{\lstinline[style=bash]{./install-sdl-gfx.sh}}

      \item Build the SOL compiler as \texttt{sol.native} and the static graphic libarary as \texttt{predefined.o}\\
      \colorbox{lightgray}{\lstinline[style=bash]{make all}}

    \end{enumerate}

  \section{SOL Quick Tour}
  This section provides a quick tour on the data types, data structures and the shape oriented programming paradigm provided by SOL, with examples. Towards the end we walk through the steps to generate an executable SOL program. For a detailed explanation of language syntax, semantics and built-int functions refer to chapter \ref{lrm}.

    \subsection{Primitive data types}
      All variables of primitive data types are declared starting with their type followed by an identification. SOL supports following primitives:
      \begin{enumerate}
        \itemsep 0em
        \item \texttt{int} - integers
        \item \texttt{float} - double precision floating point number
        \item \texttt{char} - single character from \texttt{ASCII} character set
        \item \texttt{string} - a sequence of one or more \texttt{ASCII} characters
      \end{enumerate}
      Example:\\
      \begin{lstlisting}[style=sol]
        func main() {
            int x;  /* declare primitives */
            float y;
            char c;
            string s;

            x = 5;  /* assign values to primitives */
            y = 6.5;
            c = 'a';
            s = "better call SOL";
        }
      \end{lstlisting}

    \subsection{Arrays and Shapes}
      SOL supports two complex data structures:
      \subsubsection{Array}
        An array is a fixed size sequence of primitives. Individual elements can be accessed within array by specifying indices as integers in range \texttt{0} to \texttt{1 less than array length}.\\
        Example:\\
        \begin{lstlisting}[style=sol]
          func main() {
              int [3]a;      /* declare array of 3 integers */
              a = [1, 2, 3]; /* init array with 3 elements*/
              a[1] = 4;      /* update second element value to 4 */
              a[2] = a[0];   /* last element equal to first element */
          }
        \end{lstlisting}

      \subsubsection{Shape}
        A collection of one or more variables of \textit{primitives, arrays and/or shapes} which come together to describe a \textit{shape}. These variables are called \textit{member variables} of a shape. A shape definition can also contain function definitions, referred as \textit{member functions}.\\
        Example:\\
        \begin{lstlisting}[style=sol]
          class Line {
              string name;  /* identify a line by name */
              int [2]start; /* first end point of a line */
              int [2]end;   /* other end point of a line */

              /* compulsory constructor for a shape */
              construct(int [2]s, int[2]e) {
                  /* constructor can be empty definition */
                  start = s;
                  end = e;
              }

              /* compulsory draw member function.*/
              draw() {
                  /* accepts no arguments */
                  /* can be empty definition */
              }
          }

          func main() {
              Line l; /* declare a variable of shape Line */
              l = shape Line([1,1], [2,2]); /* instantiate a Line object */
          }
        \end{lstlisting}

    \subsection{Operators}
    SOL supports \texit{arithmetic operations, relational comparisons and logical operations}. For all binary operators described in this section, the operands are specified to the left and right, respectively, of the operator and \textit{both operands must be expressions of same data types}.\\
    All logical operators accept \textit{boolean logic expressions},represented as \textit{integer expression} in SOL. Non-zero expressions correspond to \texttt{true} and a \texttt{zero} corresponds to \texttt{false}. 

    \begin{enumerate}
      \itemsep 0em
      \item \textit{Binary arithmetic operators}: \texttt{+, -, *, / \%}
      \item \textit{Relational operators (binary)}: \texttt{==, |=, >, >=, <, <=}
      \item \textit{Logical operators}: \texttt{\&\&, ||} \textit{(operands must be of type int)}
      \item \textit{Logical not}: \texttt{!} \textit{(unary and right associative)}
      \item \textit{Unary negation}: \texttt{-} \textit{(unary and right associative)}
    \end{enumerate}

    All expressions are evaluated left to right. Please refer to section \ref{precedence} for exact order of preference and associativity rules for each operator.\\

    Example:\\
    \begin{lstlisting}[style=sol]
        func main() {
            int x;
            int y;
            int c;
            float f;
            float g;

            x = 2;
            y = 3;
            f = 3.0;
            g = 6.0;

            y = x + y;  /*5: integer addition */
            x = y - x;  /*3: subtraction */

            g = g * f;  /* 18: floating point multiplication*/
            f = g / f;  /* 6: floating point division */
            y = y % 2;  /*1: modulo operation */

            c = g == f; /*0: EQUALITY is false */
            c = g != f; /*1: NOT EQUAL is true */
            c = g > f;  /*1: g GREATER THAN f */
            c = y >= x; /*1: y GREATER THAN OR EQUAL x is true */
            c = y < x;  /*0: y LESS THAN x is false */
            c = 5 <= 5; /*1: 5 LESS THAN OR EQUALS 5 is true */

            c = 5 && 0; /*0: LOGICAL AND of true and false */
            c = 2 || 0; /*1: LOGICAL OR of true and false */

            c = !c;   /*0: LOGICAL NOT of true(1) */
            f = -f;   /*-6: unary negation of arithmetic expression */
        }
    \end{lstlisting} 

    \subsection{Control flow statements}
    SOL program statements are executed in order, with the entry point being the main function. However, sometimes developers need to execute only a branch of source code (jump through some statements) or execute a portion of code repeatedly. SOL provides two control flow statements if and while for conditional branching and looping through a portion of code, respectively.

      \subsubsection{if-statement}
      An \textit {if statement block} allows to execute or skip a \texit{code branch} based on a \textit{logical expression}.\\
      Example:\\
      \begin{lstlisting}[style=sol]
        func main() {
          if (1 > 2) {
            /* condition if false; skip this code block */
            consolePrint("NOT PRINTED");
          }
          consolePrint("Hello World");
        }
      \end{lstlisting}

      \subsubsection{while-statement}
      A \texit{while statement block} allows to execute a portion of source code repeatedly until a logical condition holds \texttt{true}.
      \begin{lstlisting}[style=sol]
        func main() {
            int i;
            i = 1;
            while (i <= 5) {
                consolePrint("Hello");  /* this loop prints Hello 5 times*/
                i = i + 1;    /* loop terminates when i exceeds 5 */
            }
        }
      \end{lstlisting}

    \subsection{Functions in SOL}
    In SOL functions can be defined as a way to abstract away and re-use a code lock, with a named representation. These are useful if a particular piece of code needs to be executed at multiple places in the program. Functions in SOL can be defined as stand alone functions or as \textit{member functions} of a \textit{shape} definition. Function definitions begin with the \texttt{func} keyword and they optionally accept arguments as input values and return a result, for which the result type needs to be indicated in the function definition. A function may not return any result (void type) in which case no return type needs to be mentioned during function definition. Please refer section \ref{function} for a detailed explanation of function definition syntax.\\
    
    Example:\\
    \begin{lstlisting}[style=sol]
      /* define a function that accepts two integers and returns their sum */
      func int add(int x, int y) {
          /* sum of two integers */
          return x + y;
      }

      func main() {
          int sum;
          sum = add(2, 3);
          if (sum == 5) {
              consolePrint("CORRECT");
          }
      }
    \end{lstlisting}

    Functions in SOL also support recursion. SOL also provides a number a of type conversion functions and built-in functions for displaying and animating shapes on screen and printing text on screen or console. Please refer to section \ref{internal} for detailed information, syntax on these functions.

  \section{Rotate a Line around midpoint (Sample SOL program)}
  The following program shows a simple example to define a line, create it and rotate it around its midpoint.
  \begin{enumerate}
    \itemsep 0em
    \item As SOL treats all animation as an interaction of \textit{shapes}, we first define a straight line as a \textit{shape} in SOL, define its \texttt{draw} function.\\
      \begin{lstlisting}[style=sol]
        shape Line {
            int [2]start;
            int [3]mid;
            int [2]end;

            construct(int [2]s, int [2]e) {
                start = s;
                end = e;
                mid[0] = (s[0] + e[0]) / 2;
                mid[1] = (s[1] + e[1]) / 2;
            }

          draw() {
            /* draw red line */
            drawCurve(start, mid, end, 100, [200, 0, 0]);
          }
        }
      \end{lstlisting}

    \item The next step is to create a line object and instantiate it within \texttt{main} function, which is the entry point to a SOL program.\\
      \begin{lstlisting}[style=sol]
        func main() {
          Line l; /* declare a line shape */

          /* create a line with given end points */
          l = shape Line([10,10], [100, 100]);

          /* rotate line here */
        }
      \end{lstlisting}
  \end{enumerate}

\chapter{Language Reference Manual} \label{lrm}

  %% include the common content for lrm
  
\section{Introduction}
%% Text of abstract
SOL is a simple language that allows programmers to create 2D animations with ease. Programmers will have the ability to define and create objects, known as shapes, and dictate where they appear, and how they move. SOL uses \textit{point} (visual dot) and \textit{B\'ezier Curves} with three control points as the basic drawing controls provided to the programmer for defining shapes. As a lightweight object-oriented language, SOL allows for unlimited design opportunities and eases the burden of animation. In addition, SOL’s simplicity saves programmers the trouble of learning complicated third-party animation tools, without sacrificing control over behavior of objects.
\par

\section{Conventions}
    The following conventions are followed throughout this SOL Reference Manual.

    \begin{enumerate}
        \itemsep0em
        \item \texttt{literal} - Fixed space font for literals such as commands, functions,\\
        \hspace*{4.4em} keywords, and programming language structures.
        
        \item \textit{variable} - Italics for variables, words, and concepts being defined.
    \end{enumerate}

    The following conventions are applied while drawing and animating objects, using internal functions (see Section \ref{internal}):

    \begin{enumerate}
        \itemsep0em
        \item The origin of the drawing canvas is on the top left of the screen.
        \item The positive X-axis goes from left to right.
        \item The positive Y-axis goes from top to bottom.
        \item Positive angles specify rotation in a clockwise direction.
        \item Coordinates are specified as integer arrays of size 2, consisting of an X-coordinate followed by a Y-coordinate.
        \item Colors are specified as integer arrays of size 3, consisting of Red, Green and Blue values in the range 0 - 255, where \texttt{[0, 0, 0]} is black and \texttt{[255, 255, 255]} is white.
    \end{enumerate}

\section{Lexical Conventions}

This section describes the complete lexical conventions followed for a syntactically-correct SOL program, forming various parts of the language.

    \subsection{\textit{Comments}}
    Comments in SOL start with character sequence \texttt{/*} and end at character sequence \texttt{*/}. They may extend over multiple lines and all characters following \texttt{/*} are ignored until an ending \texttt{*/} is encountered.

    \subsection{\textit{Identifiers}}\label{identifer}
    In SOL, an identifier is a sequence of characters from the set of English alphabets, Arabic numerals and underscores (\_). The first character of an identifier should always be a lower case English alphabet. Identifiers are case sensitive. Identifiers cannot be any of the reserved keywords mentioned in section \ref{keywords}.

    \subsection{\textit{Keywords}} \label{keywords}
    Keywords in SOL include data types, built-in functions, and control statements, and may not be used as identifiers (as they are reserved).

        \begin{center}
            \begin{tabular}{ |c|c|c|c| } 
            \hline
                int     & if            & main          & shape   \\ 
                float   & while         & setFramerate  & draw	  \\ 
                char    & func          & getFramerate  & \\
                string  & construct     & print         & \\
                        & return        & consolePrint  & \\
                        &               & intToString   & \\
                        &               & floatToString & \\
                        &               & charToString  & \\
                        &               & intToFloat    & \\
                        &               & floatToInt    & \\
                        &               & render        & \\
                        &               & drawPoint     & \\
                        &               & drawCurve     & \\
                        &               & translate     & \\
                        &               & rotate        & \\
                        &               & sine			& \\
                        &               & cosine		& \\
                        &               & round			& \\

            \hline
            \end{tabular}
        \end{center}

    \subsection{\textit{Integer Constants}}
    A sequence of one or more digits representing a number in base-10, optionally preceded by a unary negation operator (\texttt{-}), to represent negative integers.\\
    Example: \texttt{1234}

    \subsection{\textit{Float Constants}}
    Similar to an integer, a float has an \textit{integer}, a decimal point (\texttt{.}), and a fractional part. Both the integer and fractional part are a sequence of one or more digits. A negative float is represented by a preceding unary negation operator (\texttt{-}).\\
    Example: \texttt{0.55  10.2}

    \subsection{\textit{Character Constants}}
    An ASCII character within single quotation marks.\\
    Example: \texttt{'x' 'a'}

    \subsection{\textit{Escape Sequences}}
    The following are special characters represented by escape sequences.
        \begin{center}
            \begin{tabular}{ |c|c| }
            \hline
                \textbf{Name}   & \textbf{Escape}\\
                \hline
                newline         & \textbackslash n\\
                tab             & \textbackslash t\\
                backslash       & \textbackslash \textbackslash\\
                single quote    & \textbackslash '\\
                double quote    & \textbackslash "\\
                ASCII NUL character & \textbackslash 0\\
            \hline
            \end{tabular}
        \end{center}

    \subsection{\textit{String constants}}
    A SOL \textit{string} is a sequence of zero or more \textit{characters} within double quotation marks.\\
    Example: \texttt{"cat"}

    \subsection{\textit{Operators}}
    SOL has mainly four categories of operators defined below:

        \subsubsection{\textit{Assignment Operator}}
        The right associative \textit{assignment operator} is denoted by the (\texttt{=}) symbol having a variable identifier to its left and a valid expression on its right. The \textit{assignment operator} assigns the evaluated value of expression on the right to the variable on the left.

        \subsubsection{\textit{Unary Negation Operator}} \label{negation}
        The right associative unary negation operator (\texttt{-}) can be used to negate the value of an arithmetic expression.

        \subsubsection{\textit{Arithmetic Operators}}
        The following table describes \texttt{binary arithmetic operators} supported in SOL which operate on two \texttt{arithmetic expressions} specified before and after the operator respectively. The said expressions must both be of type \texttt{int} or \texttt{float}. Please refer to section \ref{precedence} for precedence and associativity rules.

        \begin{center}
            \begin{tabular}{ |c|c| }
            \hline
                \textbf{Operator} & \textbf{Definition} \\
                \hline
                +   &   Addition\\
                -   &   Subtraction\\
                $*$   &   Multiplication\\
                /   &   Division\\
                \%  &   Modulo\\
            \hline
            \end{tabular}
        \end{center}

        \subsubsection{\textit{Comparison Operators}}
        The comparison operators are left-associative binary operators for comparing values of operands defined as expressions. Please refer to section \ref{precedence} for precedence and associativity rules.
        \begin{center}
            \begin{tabular}{ |c|c| }
            \hline
                \textbf{Operator} & \textbf{Definition} \\
                \hline
                ==              & Equality \\
                !=              & Not Equals \\
                \textless       & Less than \\
                \textgreater    & Greater than \\
                \textless=      & Less than or equals \\
                \textgreater=   & Greater than or equals \\
            \hline
            \end{tabular}
        \end{center}
        
        \subsubsection{\textit{Logical Operators}}
        The logical operators evaluate boolean expressions and return an integer as result: with 0 as \textit{False} and 1 as \textit{True}. Please refer to section \ref{precedence} for precedence and associativity rules.
        \begin{center}
            \begin{tabular}{ |c|c| }
                \hline
                \textbf{Operator} & \textbf{Definition}\\
                \hline
                \&\&  & AND\\
                $||$  & OR\\
                !  & NOT\\
                \hline
            \end{tabular}
        \end{center}

    \subsection{\textit{Punctuators}}
    The following symbols are used for semantic organization in SOL:
     \begin{center}
        \begin{tabular}{ |p{0.25\hsize}|p{0.75\hsize}| }
            \hline
            \textbf{Punctuator} & \textbf{Usage} \\
            \hline
            \{\}            & Used to denote a block of code. Must be present as a pair. \\
            ()              & Specifies conditions for statements before the subsequent code, or denotes the arguments of a function. Must be present as a pair. \\
            \lbrack\rbrack  & Indicates an array. Must be present as a pair. \\
            ;               & Signals the end of a line of code. \\
            ,               & Used to separate arguments for a function, or elements in an array definition. \\
            \hline
        \end{tabular}
     \end{center}
    
\section{Identifier Scope}
    \subsection{\textit{Block Scope}}
    Identifier scope is a specific area of code wherein an identifier exists. A scope of an identifier is from its declaration until the end of the code block within which it is declared.
    
    \subsection{\textit{File Scope}}
    Any identifier (such as a variable or a function) that is defined outside a code block has file scope i.e. it exists throughout the file.\\
    If an identifier with file scope has the same name as an identifier with block scope, the block-scope identifier gets precedence.

\section{Expressions and Operators}
    \subsection{\textit{Precedence and Associativity}} \label{precedence}
    SOL expressions are evaluated with the following rules:
    \begin{enumerate}
        \itemsep0em
        \item Expressions are evaluated from left to right as operators are left-associative, unless stated otherwise.

        \item Expressions within parenthesis take highest precedence and are evaluated prior to substituting in outer expression.
        
        \item The unary negation operator (\texttt{-}) and logical not operator (\texttt{!}) are placed at the second level of precedence, above the binary, comparison and logical operators. It groups right to left as described in section \ref{negation}.
        
        \item The third level of precedence is taken by multiplication (\texttt{*}), division (\texttt{/}) and modulo (\texttt{\%}) operations.
        
        \item Addition (\texttt{+}) and subtraction (\texttt{-}) operations are at the fourth level of precedence.
        
        \item At the fifth level of precedence are the comparison operators: \texttt{\textless, \textgreater, \textless=, \textgreater=}.

        \item At sixth level of precedence are the equality comparison operators: \texttt{==} and \texttt{!=}.

        \item The logical operators, OR (\texttt{||}) and AND (\texttt{\&\&}) take up the next level of precedence.

        \item At the final level of precedence, the right associative assignment operator (\texttt{=}) is placed, which ensures that the expression to its right is evaluated before assigning to the left variable identifier.

    \end{enumerate}
    
    \subsection{\textit{Dot Accessor}}
    To access members of a declared \texttt{shape} (further described in section \ref{classes}), use the dot accessor `.' followed by the identifier name of the \textit{member variable} or \textit{member function}. \\
    Example: \texttt{shape\_object.point1 /* This accesses the variable point1 within the object shape\_object */}


\section{Declaring Identifers}

    Declarations determine how an identifier should be interpreted by the compiler. A declaration should include the identifier type and the given name.

    \subsection{\textit{Type Specifiers}} \label{type}
    SOL provides four type specifiers for data types:
    \begin{itemize}
        \itemsep0em
        \item \textit{int} - integer number
        \item \textit{float} - floating-point number
        \item \textit{char} - a single character
        \item \textit{string} - string (ordered sequence of characters)
    \end{itemize}

    \subsection{\textit{Declaring Variables}}
    An identifier, also referred to as a \textit{variable}, is declared by specifying the \texttt{primitive type} or name of the \texttt{Shape}, followed by a valid identifier, as specified in section \ref{identifer}. Variables can be declared only at the beginning of a function or at the top of the source files, as global variables, which are accessible within all subsequent function or shape definitions.

    \subsection{\textit{Array Declarators}} \label{array}
    An array may be formed from any of the primitive types and \texttt{shape}s, but each array may only contain one type of primitive or \texttt{shape}. At declaration, the type specifier and the size of the array must be indicated. The array size need not be specified for strings, which are character arrays. SOL supports \texttt{fixed size arrays}, declared at compile time i.e. a program can not allocate dynamically sized arrays at runtime. Arrays are commonly used in SOL to specify coordinates with two integers and specify drawing colors in RGB format with an array of three integers.\\
    Example: \texttt{int[2] coor; /* Array of two integers */}

    \subsection{\textit{Function Declarators and Definition}} \label{function}
    Functions are declared with the keyword: \texttt{func}. This is followed by the \textit{return type} of the function. If no return type is specified, then the function automatically does not return any value. Functions are given a name (a valid \textit{identifier}) followed by function formal arguments. These arguments are a comma-separated list of variable declarations within parentheses. Primitives are passed into functions by value, and objects and arrays are passed by reference. This function declaration is then followed by the function definition, within curly braces; functions must always be defined immediately after they are declared.\\
    Functions can also be defined within \textit{shape} definitions in which case they are referred as \textit{member functions} of the \textit{shape}. (see section \ref{classes})\\

    Example:\\
    \begin{lstlisting}[style=sol]
        func example(int a, int b){
            /* a function named example that takes
               two arguments, both of type int */
        }
    \end{lstlisting}

    \subsection{\textit{Constructor Declarators}}
    Constructors are declared with the keyword: \texttt{construct}. Constructor definitions are similar to function definitions, with three additional constraints: 
    \begin{enumerate}
        \itemsep0em
        \item Constructors are defined inside the \textit{shape} definition
        \item A constructor is defined with the \texttt{construct} keyword, followed by optional formal arguments within parenthesis as a comma-separated list of variable declarations, similar to function definitions
        \item Constructors do not have a return type specified
    \end{enumerate}
    Example:\\
    \begin{lstlisting}[style=sol]
    shape Point {
        int [2]coordinate;
        construct (int x, int y) {
            /* constructor definition */
            coordinate[0] = x;
            coordinate[1] = y;
        }
    }
    \end{lstlisting}
    Please see section \ref{classes} for defining shapes in SOL and creating shape instances.

    \subsection{\textit{Definitions}}
    A definition of a primitive type variable includes a value, assigned by the assignment operator `\texttt{=}'. For defining arrays, \texttt{rvalue} is the sequence of array literals within square brackets. \textit{Shapes} are objects which are initialized by calling the \texttt{construct}, with optional parameters (see section \ref{classes}). In SOL programs, all variables \textit{must} be \textit{declared} before assigning values.\\
    Example:\\
    \begin{lstlisting}[style=sol]
        char y;       /* declarations */
        float z;
        int [3]w;     /* array declaration */
        string s;
        Triangle t;

        y = 'b';      /* definitions */
        z = 3.4;
        w = [5, 2, 0];
        s = "cats";
        t = shape Triangle();   /* a triangle object */
    \end{lstlisting}

\section{Statements}
A statement in SOL refers to a complete instruction for a SOL program. All statements are executed in order sequential order. The four types of statements are described in detail below:\\

    \subsection{\textit{Expression Statement}}
    Expression statements are those statements that get evaluated and produce a result. This can be as simple as an assignment or a function call.\\
    Example: \texttt{x = 5; /* assign 5 to identifier x */}

    \subsection{\textit{If Statement}}
    An \textit{if} statement is a conditional statement, that is specified with the \texttt{if} keyword followed by an \textit{expression} specified within a pair of parentheses; further followed by a block of code within curly braces. The code specified within the \texttt{if} block executes if the expression evaluates to a non-zero \textit{integer}.\\
    Example:\\
    \begin{lstlisting}[style=sol]
        int x;
        x = 1;
        if (x == 1) {
            /* This code gets executed */
        }
    \end{lstlisting}

    \subsection{\textit{While Statement}}
    A \textit{while} statement specifies the looping construct in SOL. It starts with the \texttt{while} keyword, followed by an expression specified within a pair of parenthesis; this is followed by a block of code within curly braces which is executed repeatedly as long as the condition in parentheses is valid. This condition is re-evaluated before each iteration and the code within \texttt{while} block executes if the condition evaluates to a non-zero \textit{integer}. \\
    Example:\\
    \begin{lstlisting}[style=sol]
            int x;
            x = 5;
            while (x > 0) {
                /* This code gets executed 5 times */
                x = x - 1;
            }
        \end{lstlisting}

    \subsection{\textit{Return statement}}
    The return statement stops execution of a function and returns to where the function was called originally in the code. Potentially returns a value; this value must conform with the return type specified in the function declaration. If no return type was specified, a \textit{return} statement without any value specified is syntactically valid (but not compulsory).\\
    Example:\\
    \begin{lstlisting}[style=sol]
        func int sum(int x, int y) {
            /* return sum of two integers */
            return x + y;
        }
    \end{lstlisting}
    
\section{Internal Functions} \label{internal}
SOL specifies a set of required/internal functions that must be defined for specific tasks such as drawing, rendering or as an entry point to the program, described below.

    \subsection{main}
    Every SOL program must contain a \texttt{main} function as this is the entrypoint of the program. The \texttt{main} function may declare and define variables or shape objects or call other functions written in the program. The \texttt{main} function does not take inputs as SOL programs do not depend on user input.\\
    Example:\\
    \begin{lstlisting}[style=sol]
        func main() {
            /* Entry point for SOL programs */
            int x;  /* variable declaration */
            x = 1;  /* assign value */
            consolePrint("Hello World"); /* call function */
        }
    \end{lstlisting}
    \underline{Arguments}: None

    \subsection{setFramerate}
    Call \texttt{setFramerate} to specify frames per second to render on screen. The frame rate is specified as a \textit{positive integer argument} and returns \texttt{0} for success and \texttt{-1} to indicate failure. By default, frame rate is set to 30 frames per second for a SOL program.\\
    \underline{Arguments}: \texttt{rate (int)}\\
    \underline{Return}: \texttt{0 for success, -1 for failure}

    \subsection{getFramerate}
    Call \texttt{getFramerate} to get the current number of frames rendered per second as an \textit{integer}.\\
    \underline{Arguments}: \texttt{None}\\
    \underline{Return}: \texttt{frames per second (int)}

    \subsection{consolePrint}
    Prints a string to the console. Commonly used to print error messages.\\
    \underline{Arguments}: \texttt{text (string)}

    \subsection{round}
    The \texttt{round} function accepts a \texttt{float} argument and returns the value rounded to the nearest integer value, rounding halfway cases away from zero instead of to the nearest integer.\\
    \underline{Argument}: \texttt{value (float)}\\
    \underline{Return}: \texttt{result (float)}

    \subsection{Trigonometric functions}
    SOL provides two basic trignometric functions, \texttt{sine} and \texttt{cosine} which can be used to easily calculate coordinates for displaying objects or other arithmetic calculations in SOL programs.

        \subsubsection{\textit{sine}}
        The \texttt{sine} function accepts an argument of type \texttt{float}, representing an \textit{angle} in \texttt{degrees} and returns
        the \textit{sine} value.\\
        \underline{Arguments}: \texttt{angle (float)}\\
        \underline{Return}: \texttt{sine value (float)}


        \subsubsection{\textit{cosine}}
        The \texttt{cosine} function accepts an argument of type \texttt{float}, representing an \textit{angle} in \texttt{degrees} and returns
        the \textit{cosine} value.\\
        \underline{Arguments}: \texttt{angle (float)}\\
        \underline{Return}: \texttt{cosine value (float)}

    \subsection{Type Conversion Functions}
    SOL provides the following type conversion functions for converting expressions of a given type to an expression of another type.

        \subsubsection{\textit{intToString}}
        Accepts an expression (\texttt{src}) of type \texttt{int} as the argument and returns the \texttt{string} representation of evaluated result.\\
        \underline{Argument}: \texttt{src (int)}\\
        \underline{Return}: value of type \texttt{string}

        \subsubsection{\textit{floatToString}}
        Accepts an expression (\texttt{src}) of type \texttt{float} as the argument and returns the \texttt{string} representation of evaluated result.\\
        \underline{Argument}: \texttt{src (float)}\\
        \underline{Return}: value of type \texttt{string}

        \subsubsection{\textit{charToString}}
        Accepts an expression (\texttt{src}) of type \texttt{char} as the argument and returns the \texttt{string} representation of evaluated result.\\
        \underline{Argument}: \texttt{src (char)}\\
        \underline{Return}: value of type \texttt{string}

        \subsubsection{\textit{intToFloat}}
        Accepts an expression (\texttt{src}) of type \texttt{int} as the argument and returns the \texttt{float} representation of evaluated result.\\
        \underline{Argument}: \texttt{src (int)}\\
        \underline{Return}: value of type \texttt{float}

        \subsubsection{\textit{floatToInt}}
        Accepts an expression (\texttt{src}) of type \texttt{float} as the argument and returns the \texttt{int} representation of evaluated result.\\
        \underline{Argument}: \texttt{src (float)}\\
        \underline{Return}: value of type \texttt{int}

\section{Drawing Functions}
The following set of functions are also a category of internal/required functions, which describe the drawing aspects for \texttt{shape} objects defined in a SOL program.

    \subsection{\textit{drawCurve}}
    \texttt{drawCurve} is one of the basic internal functions; it is used to draw a B\'ezier curve. SOL aims to define all possible shapes as a collection of B\'ezier curves. The function arguments in order are, the \textit{three control points} for the curve, a \textit{step size} to control smoothness of the curve, and the \textit{color} of curve (in RGB format).\\
    \underline{Arguments}: \texttt{pt1 (int[2]), pt2 (int[2]), pt3 (int[2]), steps(int), color (int[3])}

    \subsection{\textit{drawPoint}}
    Draws a point at a specified coordinate in the specified color.\\
    \underline{Arguments}: \texttt{pt (int[2]), color (int[3])}

    \subsection{\textit{print}}
    Displays horizontal text on the render screen at the coordinates specified by the user, in the specified color.\\
    \underline{Arguments}: \texttt{pt (int[2]), text (string), color (int[3])}

    \subsection{\textit{draw}}
    For every \texttt{shape} definition, there must be a \texttt{draw} function defined by the programmer. The \texttt{draw} function does not accept any input arguments and is called internally to display the object on screen. The \texttt{drawCurve}, \texttt{drawPoint} and \texttt{print} functions calls can be used within the \texttt{draw} definition to describe the actual drawing of an object.\\
    At runtime \texttt{draw} functions of all objects instantiated in the \texttt{main} function are called, to render the final scene.

\section{Animation Functions} \label{animation}
The following functions are used to animate the objects drawn in a SOL program.

    \subsection{\textit{translate}}
    Displaces a \texttt{shape} across a straight path by specifying a two-element array of relative displacements - the first element is the displacement in pixels along the horizontal axis and the second element along the vertical axis - over a specified time period (in seconds).\\
    \underline{Arguments}: \texttt{displace (int[2]), time (int)}

    \subsection{\textit{rotate}}
    Rotates the position of a \texttt{shape} around a specified axis point by a specified angle (in degrees), over a time period (in seconds).\\
    \underline{Arguments}: \texttt{axis (int[2]), angle (float), time (float)}

    \subsection{\textit{render}}
    Specifies the set of motions to be animated. This code-block can be defined for shapes that need to move, or can be left undefined for static shapes. Within this block, various \texttt{rotate} and \texttt{translate} calls can be made to move the shape. This must be specified in the \texttt{main} function.\\

\section{Classes} \label{classes}
SOL follows an object-oriented paradigm for defining objects (drawn \texttt{shape}s) which can be further animated using the animation functions described in Section \ref{animation}.

    \subsection{\textit{shape}}
    Similar to a class in C++; a \textit{shape} defines a particular 2-D shape as part of the drawing on screen. The name of a \textit{shape} must always start with an uppercase English alphabet.

    \subsubsection{Shape definition}
    A \textit{shape} definition starts with the \texttt{shape} keyword, followed by the \textit{shape name},(Example: \texttt{Triangle}) and the definition within curly braces (\{\}) code block. \textit{Shape} definitions may optionally contain \textit{member variables}.\\
    
    Every \textit{shape} must define a \textit{constructor} using \texttt{construct} keyword, and a \texttt{draw} function. The \texttt{construct} definition can optionally have \textit{formal arguments} as input parameters. The \texttt{draw} function does not accept any arguments and its definition can have multiple \texttt{drawPoint}, \texttt{drawCurve} and \texttt{print} function calls to describe the screen display of the \textit{shape} object.\\

    It is possible to define \textit{functions} in a shape definition. The \textit{member functions} are defined with the same rules as specified in section \ref{function}.\\
    When \textit{member variables} are accessed within a member function, it is implied that the member variables belong to the current object that calls the function. If a \textit{member variable} or \textit{global variable} name is same as that of a \textit{local variable} or \textit{formal argument} in the function definition, then the \textit{local variable} or \textit{formal argument} overshadows the other conflicting variable.\\

    Example:\\
    \begin{lstlisting}[style=sol]
        shape Triangle {
            int[2] a; /* Corners of a triangle */
            int[2] b;
            int[2] c;

            int[2] p;  /* mid points of lines*/
            int[2] q;
            int[2] r;

            construct (int [2]x, int [2]y, int [2]z) {
                a = x;
                b = y;
                c = z;

                findCenter(p, a, b);
                findCenter(q, b, c);
                findCenter(r, a, c)
            }

            /* internal draw function definition */
            draw() {
                /* Draw triangle lines with bezier curves */
                drawcurve(a, p, b, 100, [255,0,0]); /*red*/
                drawcurve(b, q, c, 100, [0,255,0]); /*green*/
                drawcurve(c, r, a, 100, [0,0,255]); /*blue*/
            }

            /* write result in pre-allocated array res */
            func findCenter(int[2]m, int[2]x, int[2]y){
                m[0] = (x[0] + y[0]) / 2;
                m[1] = (x[1] + y[1]) / 2;
            }
        }
    \end{lstlisting}

    \subsubsection{Creating Shape Instances}
    Actual instances for a \textit{shape} definition can be created, which represent the actual shapes rendered on the screen.\\
    To instantiate an object for a shape, we first declare a variable of defined \textit{shape} (say \texttt{Triangle}) and then instantiate it by calling the \textit{constructor}.\\
    Example:\\
    \begin{lstlisting}[style=sol]
        func main() {
            /* declare variable of shape Triangle */
            Triangle t; 

            /* instantiate a triangle */
            t = shape Triangle([100,100], [200,100], [150,200]); 
        }
    \end{lstlisting}


\chapter{Project Plan}

  This chapter details the project development, testing process, timelines, milestones and roles and responsibilities of each team member.

  \section{Project Development Process}
  Throughout the SOL compiler development process we tried to stick as closely as possible to in-lecture instructions and suggestions given by instructor, Prof. Stephen A. Edwards, and our project mentor Connor Abbott. The four major phases of project development are described in the following sections.

    \subsection{Planning}
      Once the team decided on addressing easy animations through an object-oriented programming language, we quickly got down to listing as many features as we could think of with respect to a graphics animation. As Aditya Narayanamoorthy \textit{(Language Guru)} had taken a Computer Graphics course previously, he helped us pick out the most basic and necessary features for the language, such as using \textit{B\'ezier curves} to describe shapes in our language. We tried to narrow down the minimal number of language features that would allow a SOL programmer to describe, display and animate a shape on screen using SOL. In case of a disagreement on the features to be included, the final call was taken by the \textit{Language Guru}. All the decided features are detaile in the SOL Language Reference Manual (chapter \ref{lrm}), which every team member used as a reference point for the rest of the project development and testing phases.\\

      In our planning phase we also decided on the software development process, including division of work based on each member's expertise, discussed time availabilities, tools and development environments to use, summarized below:
      \begin{enumerate}
        \itemsep 0em
        \item We scheduled two days per week for 3 hour long meetings for the team to work together on the project.
        \item Chose Github as the version control system for our project, in conjunction with Slack for team communication.
      \end{enumerate}

    \subsection{Specfication}
      For specifying the exact feature details, we required a lightweight graphics library, that would interface with the graphics backend and link well against the LLVM compiler output and develop a good understanding of its working. Erik, the systems architect, tried out a few graphics library and suggested using SDL2, primarily because of its ease of use for rendering simple graphics on screen and it's bindings of multiple languages including C, C++ and OCaml.\\

      The next major specification step for SOL programming language, was to decide on how to represent and render shapes. Aditya, our Language Guru, suggested that all shapes can be expressed as a combination of B\'ezier curves and hence the basic units of drawing in SOL, available to a programmer should be the \texttt{drawPoint} and \texttt{drawCurve} functions to draw shapes, along with \texttt{translate} and \texttt{rotate} functions to animate the drawings in a 2D plane. To accomplish drawing B\'ezier curves, on top of SDL2, we chose the SDL2\_gfx library which was compiled and integrated into the development environment by Kunal and Erik.\\

      Throughout the compiler development, we asked our mentor Connor Abbott for advice on how to prioritize  and implement features crucial for SOL, like fixed size arrays v/s dynamic arrays or advantages of using an SAST over a simple AST. During the development phase we also realized that some of the features that we had originally planned for SOL were not quite straightforward to implement, or differed from our initial understanding. This made us modify some specific components of the language, which we updated in the Language Reference Manual and testing suite, after consulting with our mentor.

    \subsection{Development and Testing}
      We followed a \textit{test driven development} approach for SOL language as far as possible. Initially test cases were written based on the features mentioned in our programming language reference manual. Test cases were divided into two main categories, features that are allowed in SOL, such as integer addition, defining shapes etc and failure cases which check whether the compiler properly rejects syntactically or semantically incorrect programs. Each test case comprised of a basic SOL program source code that would test one particular feature or component of the language and compare the actual output against an standardized file containing expected output messages.\\

      The compiler was built by integrating one feature at a time, end to end, starting at the scanner, all the way up to codegen phase and which would allow the compiler to support that feature alongside previously implemented features. Each feature implementation was tested against the test suite to ensure correct behaviour and also ensure that it does not breaks any of the previously implemented features, using the testing script and Travis CI builds that were triggered on each code push to our github repository.\\

      Towards the end of project development timeline, we also incorporated a manual testing component, as there is no way to ensure correct display of graphics within Travis CI environment. These tests can be run using the same test suite script, with manual inspection of displayed graphics to ensure correctness.\\

  \section{Programming Style Guide}
    \begin{itemize}
      \itemsep 0em
      \item Most of our compiler source code is written in OCaml, we simply used the standard Ocaml programming guidelines as mentioned on the official tutorials page\\
      \colorbox{lightgray}{\href{https://ocaml.org/learn/tutorials/guidelines.html}{https://ocaml.org/learn/tutorials/guidelines.html}}

      \item To enforce OCaml style guides, Kunal set up the open sourced merlin linter and analyzer in sublime text editor\\
      \colorbox{lightgray}{\href{https://github.com/let-def/sublime-text-merlin}{https://github.com/let-def/sublime-text-merlin}}

      \item A significant component of SOL compiler is the \textit{predefined.o} static library written in C (C99 standard) for which the C++ linter was used within sublime text.\\
      \colorbox{lightgray}{\href{ https://github.com/SublimeLinter/SublimeLinter-cppcheck}{ https://github.com/SublimeLinter/SublimeLinter-cppcheck}}

      \item As a general rule of thumb, we tried to use descriptive variable names throughout our code, which helps the reader to easily understand the code.
    \end{itemize}

  \section{Project Timeline}
    \subsection{Milestones}
      \begin{center}
        {\renewcommand{\arraystretch}{1.5}
        \begin{tabular}{|p{8cm}|p{4cm}|}
        \hline
          \textbf{Milestone} & \textbf{Date} \\
          \hline
          Initial research \& brainstorming & Sept. 11 \\
          \hline
          Decide on language idea & Sept. 19 \\
          \hline
          Finish Proposal & Sept. 26 \\
          \hline
          Finish LRM & Oct. 16 \\
          \hline
          Add regression test suite/Travis CI & Oct. 29 \\
          \hline
          Implement primitives \& other basics & Nov. 1 \\
          \hline
          Hello World Demo & Nov. 8 \\
          \hline
          Implement arrays, internal functions & Dec. 2 \\
          \hline
          Implement shapes & Dec. 12 \\
          \hline
          Implement drawing (SDL) & Dec. 16-17 \\
          \hline
          Update and Finish Testing (manual) & Dec. 17-19 \\
          \hline
          Final Report Preparation & Dec. 18-19 \\
          \hline
          Demo Day & Dec. 20\\
        \hline
        \end{tabular}
      \end{center}

    \subsection{Github timeline}
      \includegraphics[scale=0.37]{github-timeline}


% Appendix code listings
\begin{appendices}

\chapter{SOL Compiler}

Code listing for compiler code. Author names are mentioned as first comment line of each code listing.

    \section{scanner.mll}
    \lstinputlisting[style=caml]{../scanner.mll}

    \section{parser.mly}
    \lstinputlisting[style=caml]{../parser.mly}

    \section{ast.ml}
    \lstinputlisting[style=caml]{../ast.ml}

    \section{semant.ml}
    \lstinputlisting[style=caml]{../semant.ml}

    \section{sast.ml}
    \lstinputlisting[style=caml]{../sast.ml}

    \section{codegen.ml}
    \lstinputlisting[style=caml]{../codegen.ml}

    \section{sol.ml}
    \lstinputlisting[style=caml]{../codegen.ml}

    \section{predefined.h}
    \lstinputlisting[style=caml]{../predefined.h}

    \section{predefined.c}
    \lstinputlisting[style=caml]{../predefined.c}

    \section{Makefile}
    \lstinputlisting[style=make]{../Makefile}


\chapter{Environment Setup}

The following scripts can be used for installing dependencies and setting up environment.

    \section{install-llvm.sh}
    \lstinputlisting[style=bash]{../install-llvm.sh}

    \section{install-sdl-gfx.sh}
    \lstinputlisting[style=bash]{../install-sdl-gfx.sh}

\chapter{Automated testing}\label{test-suite}

The first two scripts are used for automated testing on Travis CI. For individual test cases, the author names are mentioned as first line of each test case.

    \section{.travis.yml}
    \lstinputlisting[style=bash]{../.travis.yml}

    \section{testall.sh}
    \lstinputlisting[style=bash]{../testall.sh}

    %% include latest set of automated test files
    %% use g.py scrip to generate the test-case.tex
    %% recompile the latex document
    \subsection{fail-array-assign.sol}
\lstinputlisting[style=sol,aboveskip=-4pt]{../tests/fail-array-assign.sol}

\subsection{test-char-to-string.sol}
\lstinputlisting[style=sol,aboveskip=-4pt]{../tests/test-char-to-string.sol}

\subsection{fail-div-semantic.sol}
\lstinputlisting[style=sol,aboveskip=-4pt]{../tests/fail-div-semantic.sol}

\subsection{test-add.sol}
\lstinputlisting[style=sol,aboveskip=-4pt]{../tests/test-add.sol}

\subsection{test-precedence.sol}
\lstinputlisting[style=sol,aboveskip=-4pt]{../tests/test-precedence.sol}

\subsection{test-if.sol}
\lstinputlisting[style=sol,aboveskip=-4pt]{../tests/test-if.sol}

\subsection{fail-prod-semantic.sol}
\lstinputlisting[style=sol,aboveskip=-4pt]{../tests/fail-prod-semantic.sol}

\subsection{test-empty-function.sol}
\lstinputlisting[style=sol,aboveskip=-4pt]{../tests/test-empty-function.sol}

\subsection{fail-array-access-pos.sol}
\lstinputlisting[style=sol,aboveskip=-4pt]{../tests/fail-array-access-pos.sol}

\subsection{fail-parameter-floatint.sol}
\lstinputlisting[style=sol,aboveskip=-4pt]{../tests/fail-parameter-floatint.sol}

\subsection{test-while.sol}
\lstinputlisting[style=sol,aboveskip=-4pt]{../tests/test-while.sol}

\subsection{fail-return-void-int.sol}
\lstinputlisting[style=sol,aboveskip=-4pt]{../tests/fail-return-void-int.sol}

\subsection{test-product.sol}
\lstinputlisting[style=sol,aboveskip=-4pt]{../tests/test-product.sol}

\subsection{fail-array-access-neg.sol}
\lstinputlisting[style=sol,aboveskip=-4pt]{../tests/fail-array-access-neg.sol}

\subsection{test-logical.sol}
\lstinputlisting[style=sol,aboveskip=-4pt]{../tests/test-logical.sol}

\subsection{fail-return-int-string.sol}
\lstinputlisting[style=sol,aboveskip=-4pt]{../tests/fail-return-int-string.sol}

\subsection{test-array-pass-ref.sol}
\lstinputlisting[style=sol,aboveskip=-4pt]{../tests/test-array-pass-ref.sol}

\subsection{test-int-to-string.sol}
\lstinputlisting[style=sol,aboveskip=-4pt]{../tests/test-int-to-string.sol}

\subsection{test-float-to-string.sol}
\lstinputlisting[style=sol,aboveskip=-4pt]{../tests/test-float-to-string.sol}

\subsection{fail-if.sol}
\lstinputlisting[style=sol,aboveskip=-4pt]{../tests/fail-if.sol}

\subsection{fail-add-semantic.sol}
\lstinputlisting[style=sol,aboveskip=-4pt]{../tests/fail-add-semantic.sol}

\subsection{test-hello.sol}
\lstinputlisting[style=sol,aboveskip=-4pt]{../tests/test-hello.sol}

\subsection{fail-assign-stringint.sol}
\lstinputlisting[style=sol,aboveskip=-4pt]{../tests/fail-assign-stringint.sol}

\subsection{test-assign-variable.sol}
\lstinputlisting[style=sol,aboveskip=-4pt]{../tests/test-assign-variable.sol}

\subsection{test-array-assign.sol}
\lstinputlisting[style=sol,aboveskip=-4pt]{../tests/test-array-assign.sol}

\subsection{test-comparison.sol}
\lstinputlisting[style=sol,aboveskip=-4pt]{../tests/test-comparison.sol}

\subsection{fail-add-intstring.sol}
\lstinputlisting[style=sol,aboveskip=-4pt]{../tests/fail-add-intstring.sol}

\subsection{test-division.sol}
\lstinputlisting[style=sol,aboveskip=-4pt]{../tests/test-division.sol}

\subsection{test-associativity.sol}
\lstinputlisting[style=sol,aboveskip=-4pt]{../tests/test-associativity.sol}

\subsection{test-shape-define.sol}
\lstinputlisting[style=sol,aboveskip=-4pt]{../tests/test-shape-define.sol}

\subsection{test-array-access.sol}
\lstinputlisting[style=sol,aboveskip=-4pt]{../tests/test-array-access.sol}



\chapter{Manual Testing}\label{mnl-test}

  \section{mnl-composite-square.sol}
\lstinputlisting[style=sol,aboveskip=-4pt]{../tests/manual/mnl-composite-square.sol}

\section{mnl-drawpoint.sol}
\lstinputlisting[style=sol,aboveskip=-4pt]{../tests/manual/mnl-drawpoint.sol}

\section{test-triangle-translate.sol}
\lstinputlisting[style=sol,aboveskip=-4pt]{../tests/manual/test-triangle-translate.sol}

\section{mnl-hello.sol}
\lstinputlisting[style=sol,aboveskip=-4pt]{../tests/manual/mnl-hello.sol}

\section{mnl-line.sol}
\lstinputlisting[style=sol,aboveskip=-4pt]{../tests/manual/mnl-line.sol}

\section{mnl-thick-line.sol}
\lstinputlisting[style=sol,aboveskip=-4pt]{../tests/manual/mnl-thick-line.sol}

\section{mnl-triangle.sol}
\lstinputlisting[style=sol,aboveskip=-4pt]{../tests/manual/mnl-triangle.sol}



\end{appendices}

\end{document}
